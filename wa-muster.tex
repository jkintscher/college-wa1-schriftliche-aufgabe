\documentclass{wa}

\begin{document}

\maketitle{Jan von Bargen}{Informatik}
	{Joschka Kintscher}{Digitale Medien}
	{Janine Thönsing}{Digitale Medien}{24.10.2013}{1}

% ------------------------------------------------------------------------------
% Aufgabe A
% ------------------------------------------------------------------------------

\section{Quellensuche}

  \begin{enumerate}
    \item\textbf{new media \& society}
    \newline
    Standort: Zentrale/E4, Regal: z puz, Buch: jc/313
    \newline
    \cite{vergeer2013}

    \item\textbf{Wirtschaftsinformatik 3/2006}
    \newline
    Zentrale/E2, Regal: z kyb, Buch: 400 j/962
    \newline
    \cite{thiesse2006}

    \item\textbf{Informatik-Spektrum}
    \newline
    \cite{lintu2009}

    \item\textbf{FIfF Kommunikation}
    \newline
    \cite{bockerman2013}

    \item\textbf{Lecture Notes in CS}
    \newline
    \cite{temdee2006}

    \item\textbf{Film über Joseph Weizenbaum}
    \newline
    Zentrale/E4 Mediathek, Film: ph0858
    \newline
    \cite{haas2006}
    \newline
  \end{enumerate}

% ------------------------------------------------------------------------------
% Aufgabe B
% ------------------------------------------------------------------------------

\section{LaTeX}

  Das Buch "LaTeX: Basissystem, Layout, Formelsatz" \cite{braune2006} liefert viele Programmierbeispiele, die durchweg anschaulich mit Screenshots illustriert und von farblich formatierten Syntaxbeispielen begleitet werden. Der Inhalt folgt einer sehr klaren und einfach zu verstehenden Struktur. Anders als viele andere Fachliteratur ist das gesamte Werk in deutscher Sprache veröffentlicht und macht es so vielen Einsteigern deutlich einfacher. Andere Einführungsbücher, wie z.B. "Latex Band 1: Einführung" machen deutlich weniger Syntax- und Anwendungsbeispiele und erlauben es auf Grund ihres sehr technischen Aufbaus weniger, einen einfachen Einstieg in die Thematik zu finden.
  \newline
  \cite{haas2006}
  \newline
\end{enumerate}

\begin{LARGE}
  \textbf{Aufgabe b)}
\end{LARGE}

\bibliography{literatur}
\bibliographystyle{alpha}

\end{document}
